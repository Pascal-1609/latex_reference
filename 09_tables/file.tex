\documentclass{article}

\usepackage{siunitx}
\usepackage{multirow}
% for prettier tables
\usepackage{booktabs}
% for displaying tables on multiple pages
\usepackage{longtable}
% to display tables in landscape
\usepackage{rotating}
% for displaying long cells
\usepackage{tabularx}

\title{My first document}
\date{2020-07-03}
\author{Christopher}

\sisetup
{
	% rounds numbers...
	round-mode		= places,
	% ...to 2 places
	round-precision	= 2,
}

\begin{document}
	
	%             <- float setting
	\begin{table}[h!]
	\begin{center}
		% the \caption and \label commands can be used in the same way as for pictures
		\caption{Your first table.}
		\label{tab:table1}
		% alignments: 1st column left, 2nd middle, 3rd right, with vertical lines in between
		\begin{tabular}{l|c|r}
			\textbf{Value 1} & \textbf{Value 2} & \textbf{Value 3}\\
			$\alpha$         & $\beta$          & $\gamma$ \\
			% horizontal line
			\hline
			% lines (doesn't have to be aligned but easier to read)
			1                & 1110.1           & a \\
			2                & 10.1             & b \\
			3                & 23.113231        & c \\
		\end{tabular}
	\end{center}
	\end{table}
	
	\begin{table}[ht]
	\begin{center}
		\caption{Table with aligned numbers.}
		\label{tab:table2}
		% S is can be used if the package siunitx is used -> numbers get alligned and rounded
		\begin{tabular}{l|S|r}
			\textbf{Value 1} & \textbf{Value 2} & \textbf{Value 3}\\
			$\alpha$         & $\beta$          & $\gamma$ \\
			\hline
			1                & 1110.1           & a \\
			2                & 10.1             & b \\
			3                & 23.113231        & c \\
		\end{tabular}
	\end{center}
	\end{table}
	
	\begin{table}[ht]
	\begin{center}
		\caption{Table with multirow.}
		\label{tab:table3}
		\begin{tabular}{l|S|r}
			\textbf{Value 1}      & \textbf{Value 2} & \textbf{Value 3}\\
			$\alpha$              & $\beta$          & $\gamma$ \\
			\hline
			%         <- number of rows
			%            <- width (* -> auto)
			%               <- content
			\multirow{2}{*}{1, 2} & 110.1            & a \\
			% first row is omitted
			                      & 10.1             & b \\
			\hline
			3                     & 23.113231        & c \\
		\end{tabular}
	\end{center}
	\end{table}
	
	\begin{table}[ht]
	\begin{center}
		\caption{Table with multicolumn.}
		\label{tab:table4}
		\begin{tabular}{l|S|r}
			\textbf{Value 1} & \textbf{Value 2} & \textbf{Value 3}\\
			$\alpha$         & $\beta$          & $\gamma$ \\
			\hline
			%            <- number of columns
			%               <- alignment
			%                  <- content
			\multicolumn{2}{c|}{1, 2}          & a \\
			\hline
			2               & 10.1             & b \\
			3               & 23.113231        & c \\
		\end{tabular}
	\end{center}
	\end{table}
	
	\begin{table}[ht]
	\begin{center}
		\caption{Table with multicolumn and multirow.}
		\label{tab:table5}
		\begin{tabular}{l|S|r}
			\textbf{Value 1} & \textbf{Value 2}        & \textbf{Value 3}\\
			$\alpha$         & $\beta$                 & $\gamma$ \\
			\hline
			% multicolumn spanning 2 columns with a multirow spanning 2 rows
			\multicolumn{2}{c|}{\multirow{2}{*}{1234}} & a \\
			% empty placeholder
			\multicolumn{2}{c|}{}                      & b \\
			\hline
			3                & 23.113231               & c \\
			4                & 23.113231               & d \\
		\end{tabular}
	\end{center}
	\end{table}
	
	\begin{table}[ht]
	\begin{center}
		\caption{Table with booktabs.}
		\label{tab:table6}
		\begin{tabular}{l|S|r}
			% by booktabs
			\toprule
			\textbf{Value 1} & \textbf{Value 2} & \textbf{Value 3}\\
			$\alpha$         & $\beta$          & $\gamma$ \\
			% by booktabs
			\midrule
			1                & 1110.1           & a \\
			2                & 10.1             & b \\
			3                & 23.113231        & c \\
			% by booktabs
			\bottomrule
		\end{tabular}
	\end{center}
	\end{table}
	
	% replaces \begin{table}, alignment must be specified here (no more tabular)
	\begin{longtable}[c]{l|S|r}
	
		\caption{Multipage table.}
		\label{tab:table7} \\
		\toprule
		\textbf{Value 1} & \textbf{Value 2} & \textbf{Value 3}\\
		$\alpha$         & $\beta$          & $\gamma$ \\
		\midrule
		% this denotes the end of the header -> only on the first page
		\endfirsthead
		
		\toprule
		\textbf{Value 1} & \textbf{Value 2} & \textbf{Value 3}\\
		$\alpha$         & $\beta$          & $\gamma$ \\
		\midrule
		% everything between \endfirsthead and \endhead will be shown as a header on every page
		\endhead
		1 & 1110.1 & a\\
		1 & 1110.1 & a\\
		1 & 1110.1 & a\\
		1 & 1110.1 & a\\
		1 & 1110.1 & a\\
		1 & 1110.1 & a\\
		1 & 1110.1 & a\\
		1 & 1110.1 & a\\
		1 & 1110.1 & a\\
		1 & 1110.1 & a\\
		1 & 1110.1 & a\\
		1 & 1110.1 & a\\
		1 & 1110.1 & a\\
		1 & 1110.1 & a\\
		1 & 1110.1 & a\\
		1 & 1110.1 & a\\
		1 & 1110.1 & a\\
		1 & 1110.1 & a\\
		1 & 1110.1 & a\\
		1 & 1110.1 & a\\
		1 & 1110.1 & a\\
		1 & 1110.1 & a\\
		1 & 1110.1 & a\\
		3 & 23.113231 & c\\
		\bottomrule
	\end{longtable}
	
	\begin{sidewaystable}[ht]
	\begin{center}
		\caption{Sideways table.}
		\label{tab:table7}
		\begin{tabular}{l|S|r}
			% by booktabs
			\toprule
			\textbf{Value 1} & \textbf{Value 2} & \textbf{Value 3}\\
			$\alpha$         & $\beta$          & $\gamma$ \\
			% by booktabs
			\midrule
			1                & 1110.1           & a \\
			2                & 10.1             & b \\
			3                & 23.113231        & c \\
			% by booktabs
			\bottomrule
		\end{tabular}
	\end{center}
	\end{sidewaystable}
	
	\begin{table}[h!]
	\begin{tabularx}{\textwidth}{|X|X|X|}

		\textbf{thing1} & \textbf{thing2} \\
		\hline
		thing1 & Lorem ipsum dolor sit amet, consectetur adipiscing elit. Aenean vel cursus metus. Suspendisse sit amet elit sed nunc commodo interdum. Donec quis est vulputate, vulputate tellus eget, consequat urna. Quisque quis nulla fermentum, finibus sem id, egestas tortor. Aliquam sed viverra nulla. Donec scelerisque eros at orci egestas, ut dignissim purus tincidunt. Maecenas nec orci vel turpis venenatis rutrum nec in enim. Maecenas ultricies facilisis purus, eget interdum justo aliquam eu. Phasellus enim augue, pretium sed molestie non, tincidunt vel elit. Sed auctor, nisi quis laoreet rhoncus, nunc metus bibendum lectus, suscipit viverra metus libero in nunc. Nunc bibendum luctus nisi, non venenatis velit faucibus dictum. \\
		\hline
		thing2 & stuff \\

	\end{tabularx}
	\end{table}
	
\end{document}
